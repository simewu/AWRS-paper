\section{Preliminaries: Mining, Fork, and Mining Pool}
\label{sec:preliminaries}
%The first block in a blockchain is called the genesis block and then every block proceeding it is build on top of the genesis.
Each block in a blockchain contains a unique block header, where the hash of this header is used to identify the block. %The block header is 80-byte string consisting of a 4-byte Bitcoin version number, 32-byte previous block hash, 32-byte merkle root, 4-byte timestamp, 4-byte difficulty target, and a 4-byte nonce.
The mining process based on the computational-puzzle-based proof of work (PoW) generates new blocks on the distributed blockchain,
which results in the miner of the new block earning financial rewards.
For a miner to find the next block in the blockchain, the miner needs to solve a computational puzzle
which involves hashing the block header with varying nonces; 
the solution (and the new block) corresponds to finding a nonce 
such that the hash of the block header is less than a 256-bit threshold value, called the mining difficulty. The blockchain protocol adjusts the mining difficulty to scale with the advancement in computing, so that the mining of a new block occurs regularly in expectation, e.g., every ten minutes for Bitcoin.  %This difficulty is automatically adjusted frequently by the Bitcoin protocol in such a way that it will take roughly 10 minutes on average to find a valid block.
In blockchain, when two miners find and submit valid blocks almost simultaneously, the blockchain network can disagree in the latest block because of the delay in networking and synchronization, which results in a \emph{fork}. % since the Bitcoin network is then split into multiple paths.
The fork is resolved when a new block is mined on top of either of the forked blocks; 
according to the protocol, the miners (including those which were mining on the other forked block) select the longest chain and update the block header for future mining. 
%The chance that a block has a higher probability of being selected is determined by what majority of the network has already selected that block as the main block. 
%A block is selected in the main chain only when majority of the network accepts that block.
%%But, this leads to a popular attack, known as 51\% attack, where when any node in the network achieves a computing power that is greater than 50\% and gain full control over the network by being able to choose new transactions and invalidate old transactions.

Since solo-mining has high reward variance, 
miners form a \emph{mining pool} to pool/aggregate their computational resources and the corresponding reward winnings. 
%forming a mining pool, solo-miners aggregate their computational power to make the reward variance low. 
%But, in this scenario, a proper method is needed to determine the computational power of each miner such that the reward for mining a block can be distributed evenly to the miners in a pool.
Because the mining pool distributes the reward according to the individual members' contributions in the PoW, it requires a mechanisms to keep track of the mining powers and contributions of each miner.
%Here, the pool manager is responsible for ensuring the correct payment to the miners and submitting the valid block to the network. 
%The method to do this is by using shares. The pool manager selects a block difficulty d $<$ D. hence, d is easier to compute than D. Any block that is mathematically less than d do not qualify as full block but is considered as share. The chance of generating a share (PPoW) is much greater than the chance of generating a valid block (FPoW). This is important because, it provides a way for the pool manager to measure the computational power of each of its miners, by keeping track of how many shares each miner submits.
For such mechanisms, the mining pool uses \emph{shares}, which corresponds to the same PoW computational puzzle but with lower difficulty (higher threshold value) than that of the mining difficulty.
Therefore, a share solution occurs more frequently than a block solution (providing a more accurate estimation of each miners' computation contributions), and a block solution is implicitly a share solution (a nonce solution for a block is also a solution for a share). 